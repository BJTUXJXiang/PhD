
\subsection{Contributions}

\begin{frame}{Résumé des contributions}
    \begin{columns}
        \begin{column}{0.55\linewidth}
            \begin{itemize}
                \item \textbf{Correction de l'odométrie} :
                    \begin{itemize}
                        \item Régression non paramétrique + mesure externe
                        \item Modèle linéaire + optimisation boite noire
                        \item Comparaison proprioception et prédiction
                        \item Comparaison surfaces et stabilisation
                    \end{itemize}
                \item \textbf{Correction du modèle de caméra}
                \item \textbf{Synthèse de mouvements} (en cours) :
                    \begin{itemize}
                        \item Tirs
                        \item Génération par optimisation
                        \item Simulateur physique
                    \end{itemize}
                \item \textbf{Contributions logicielles} :
                    \begin{itemize}
                        \item Ingénierie logicielle ($\approx80000$ lignes)
                        \item Générateurs de marche
                        \item Utilisé par la communauté
                    \end{itemize}
            \end{itemize}
        \end{column}
        \begin{column}{0.45\linewidth}
            \centering
            \movie[
                autostart,
                width=\linewidth, 
                height=0.76\linewidth,
                poster,
                loop
            ]{}{../video/modelKick.ogv}
            \movie[
                autostart,
                width=\linewidth, 
                height=0.56\linewidth,
                poster,
                loop
            ]{}{../video/cutKick_light.mp4}
        \end{column}
    \end{columns}
\end{frame}

\subsection{Perspectives}

\begin{frame}{Perspectives -- Odométrie}
    \begin{itemize}
        \setlength\itemsep{1.0em}
        \item Expérimentation de l'odométrie visuelle
            \begin{itemize}
                \item \customtextcolor{(Oriolo et al., 2016)} mais embarqué
                \item Comparaison sur la même plateforme
                \item Correction proprioception + odométrie visuelle
            \end{itemize}
        \item Modélisation du bruit
            \begin{itemize}
                \item Bruit du déplacement, bruit expérimental
                \item $\Rightarrow$ Maximisation de la vraissamblance
                \item Travaux en cours (\textit{German Open 2017})
            \end{itemize}
        \item Générateur de marche \textit{QuinticWalk}
            \begin{itemize}
                \item Nouvelle marche 2017 plus stable
                \item Moins de bruit
            \end{itemize}
    \end{itemize}
\end{frame}
