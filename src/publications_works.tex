
\section{Liste des publications et autres travaux\label{sec:publications_and_works}}

Cette section résume rapidement les différentes publications
auxquelles j'ai participé au cours de cette thèse 
ainsi que d'autres travaux qui me semble intéressant de mentionner.
Beaucoup des points suivants ne sont pas détaillés dans ce manuscrit
mais permettent d'apprécier le contexte et la démarche ayant guidés 
les expérimentations présentées ici.

\subsection{Publications}

\begin{itemize}
    \item \footnotesize\bibentry{ForceSencorsHumanoidsVideo2015}\normalsize\leavevmode\\\\
        Cette contribution vidéo présente succinctement les capteurs de pression 
        à base de jauges de contraintes ainsi que la méthode de stabilisation
        de la marche (voir section \ref{sec:walk_stabilization}) tous deux conçus 
        par Grégoire Passault.
        La vidéo fait également état de l'amélioration de la précision de 
        l'odométrie proprioceptive en utilisant les capteurs de pression pour
        l'estimation du pied de support (voir section \ref{sec:odometry_pressure}).\\
    \item \footnotesize\bibentry{ProjectsWorkshopHumanoids2015}\normalsize\leavevmode\\\\
        Cet article livre à l'occasion d'un workshop trois projets libres et ouverts à la communauté RoboCup.
        Le premier détail plus avant le fonctionnement des capteurs de pression de Grégoire Passault
        et livre sa conception mécanique, électronique et logicielle (\url{https://github.com/Rhoban/ForceFoot}).
        Le deuxième présente la bibliothèque logicielle \textit{RhIO} (\url{https://github.com/Rhoban/RhIO})
        développée et utilisée à la fois comme système de paramétrage, de monitorage et 
        d'interface utilisateur pour les différents modules du robot.
        L'interface en ligne de commande de type \og \textit{shell} \fg est développée
        par Grégoire Passault et Steve N'Guyen.
        Cette bibliothèque est employée sur tous les robots de l'équipe depuis 2015.
        Le troisième projet propose à la communauté l'implémentation du générateur de marche 
        \textit{IKWalk} (\url{https://github.com/Rhoban/IKWalk}) décrit à la section \ref{sec:walk}.
        Ce générateur de marche est utilisé sur nos robots humanoïdes de l'année 2014 à 2016 incluse.
        À noter que ces trois projets après de multiples contacts avec la communauté 
        ont commencé à être employés par d'autres équipes en 2017.
        Une vidéo illustrant le fonctionnement de \textit{RhIO} est visible à l'adresse :
        \url{https://youtu.be/MOizgXYENLc}\\
    \item \footnotesize\bibentry{OdometryICRA2016}\normalsize\leavevmode\\\\
        Les odométries proprioceptive et prédictive sur le robot humanoïde Sigmaban 
        sont corrigées par la méthode de régression non paramétrique LWPR.
        Les données sont acquises par un système externe de capture de mouvement.
        Deux surfaces de marche et deux mouvements de marche sont comparés.
        Le contenu de cet article est approfondi dans la section \ref{sec:odometry_lwpr}.
        Une vidéo présentant le dispositif expérimental et ses résultats est disponible
        à l'adresse : \url{https://youtu.be/9HT33KMtfLw}.\\
    \item \footnotesize\bibentry{DynabanRoboCup2016}\normalsize\leavevmode\\\\
        Exposé lors du symposium annuel de la RoboCup, cet article présente la
        ré-implémentation ouverte et libre du firmware à bord des 
        servomoteurs Dynamixel MX (\url{https://github.com/RhobanProject/Dynaban}) 
        réalisé par Rémi Fabre.
        En plus des fonctionnalités initiales du servomoteur, 
        un contrôle en couple par commande anticipée
        (\textit{feedforward}) directement embarqué en interne
        est expérimenté afin de réduire les 
        erreurs de suivi de trajectoire dans les jambes du robot Sigmaban.
        Les tests sont effectués sur un mouvement de tir en simple support très dynamique
        et généré hors ligne par optimisation en boite noire 
        (minimisation de la somme des couples, voir section \ref{sec:motion_generation}).
        Le modèle dynamique inverse du robot est utilisé pour le calcul des couples
        nécessaires à l'exécution du mouvement. 
        Les trajectoires de positions et de couples sont ensuite approximées par des
        splines cubiques puis transmises à chaque servomoteur.
        Les précisions ainsi que les retards de suivi par rapport à la consigne, induits par 
        le contrôle proportionnel avec et sans pré compensation sont comparés.\\
    \item \footnotesize\bibentry{ChampionPaperRoboCup2016}\normalsize\leavevmode\\\\
        Cet article écrit en commun avec l'équipe Rhoban fait suite à la victoire de
        l'équipe à la compétition RoboCup 2016 dans la catégorie des petits robots humanoïdes
        jouant au football.
        Différentes particularités techniques liées au succès de l'équipe sont décrites et 
        partagées avec la communauté.
        Sont évoqués, les capteurs de pression sous les pieds, la méthode de stabilisation 
        de la marche et l'architecture mécanique et électronique du robot, le choix spécifique de
        la lentille et de la caméra du robot, l'architecture du module de traitement d'image 
        et de vision, le fonctionnement du module de localisation à l'aide d'un filtre à particule, 
        nos outils de monitorage et de contrôle logiciel (RhIO), l'utilisation des différents 
        modèles géométriques prédictifs et proprioceptifs internes du robot 
        et le calcul de l'odométrie, 
        le comportement du jeu d'équipe des robots.\\
    \item \footnotesize\bibentry{ApproachICAPS2017}\normalsize\leavevmode\\\\
        Pour réduire la complexité opérationnelle posée par le système de capture de mouvement,
        une méthode de calibration de l'odométrie proprioceptive et prédictive est proposée
        ne nécessitant pas de capteur externe. Un modèle de correction linéaire est appris
        sur le robot Sigmaban au travers d'une optimisation en boite noire par l'algorithme CMA-ES.
        Cette méthode est appliquée avec succès lors de l'édition 2016 de la compétition RoboCup.
        De plus, Ludovic Hofer met en oeuvre et évalue une technique de planification de pas
        basée sur la résolution d'un problème MDP (\textit{Markovian Decision Process}) continue.
        Cette planification permet d'améliorer le temps de déplacement du robot pour se 
        placer devant une balle en position de tir.
        Ces travaux sont détaillés à la section \ref{sec:odometry_cmaes}.
        Une vidéo comparant les différents mouvements d'approche de la balle 
        est disponible à l'adresse : \url{https://youtu.be/PNA-rpNKfsY}.\\
\end{itemize}

\subsection{Autres travaux\label{sec:other_works}}

\begin{itemize}
    \item \textbf{Simulateur planaire Euler-Lagrange : }
        Développement d'un petit simulateur dynamique 2d en C++ 
        se basant sur le formalise de Euler-Lagrange.
        Les équations de la dynamique sont calculées de manière symbolique
        afin de le rendre la modélisation de la chaine cinématique générique.
        Les sources non documentées peuvent être trouvées à l'adresse
        \url{https://github.com/RhobanProject/SimLagrange}.
        Ce simulateur a été employé par Steve N'Guyen pour une étude
        des points fixes de la dynamique de plusieurs marcheurs passifs.\\
    \item \textbf{Expérimentation de la stabilité sur tapis roulant et optimisation sans modèle : }
        Le robot Sigmaban est asservit grâce à un système externe de capture
        de mouvement afin de marcher tout droit sur un tapis roulant pendant 
        de longues périodes.
        Un sous ensemble des paramètres de la marche est exploré localement 
        autour de la configuration initiale.
        Durant ses séquences, la pose du robot sur le tapis roulant ainsi que
        l'ensemble de ses capteurs sont enregistrés.
        Pour chaque jeu de paramètres, le comportement du robot 
        sur de nombreux pas est capturé.
        L'objectif étant d'une part, de construire un critère de stabilité de 
        la marche basé sur l'effort d'asservissement ainsi que les 
        mesures des gyromètres\footnote{Une estimation de la stabilité par le calcul 
        de l'application de Poincaré entre deux cycles de marche a également été envisagée. 
        Malheureusement, le bruit entachant les mesures n'a pas permit d'obtenir 
        de résultat significatif.}.
        D'autre part, une descente de gradient est menée dans le but final d'améliorer 
        la stabilité de la marche.
        Le gradient est calculé par un algorithme de différences finies 
        et de régression linéaire.
        L'objectif final de ces travaux était en réalité l'expérimentation
        des méthodes d'apprentissage par renforcement sans modèle (\textit{model-free})
        développées par \cite{peters_reinforcement_2008}, \cite{kober_policy_2009}.

        Il s'est avéré que le critère de stabilité était soumis à un bruit très important dû 
        à la fois aux nombreuses instabilités du robot ainsi qu'au processus
        d'asservissement lui même.
        Il n'a pas été possible de dégager des données un gradient significatif
        permettant une optimisation de la stabilité.
        De plus, une exploration trop forte des paramètres de la marche rendait le
        robot impossible à asservir sur le tapis roulant et entrainant rapidement sa chute.
        Il est également possible qu'au vu du bruit, les paramètres initiaux 
        de marche étaient déjà quasiment optimaux pour ce générateur.
        Par la suite, le choix a été fait de se tourner vers les méthodes 
        d'apprentissage avec modèle (\textit{model-based}), permettant 
        en théorie une meilleure utilisation des données expérimentales.

        À noter que cette expérimentation à eut lieu en 2014. 
        Étant donné que la mécanique, l'électronique et le logiciel
        (par exemple la marche\footnote{Le mouvement de marche utilisé en 2014 était basé 
        sur le générateur \textit{CartWalk}. Partageant les mêmes principes que 
        la marche \textit{IKWalk}, ce générateur était qualitativement moins stable 
        et moins paramétrable.}) ont depuis beaucoup évolué, il pourrait être intéressant 
        de retenter cette étude.
        Une vidéo du dispositif expérimental est disponible à l'adresse : 
        \url{https://youtu.be/xxazn_JJaWs}\\.
    \item \textbf{Bibliothèque de communication bas niveau \textit{RhAL} : }
        La bibliothèque \textit{RhAL} a été développée afin d'homogénéiser et
        de mieux synchroniser la communication entre le programme principal contrôlant 
        les robots (\textit{RhobanServer}) et les différents composants matériels :
        servomoteurs, centrale inertielle, capteurs de pression et boutons de contrôle.
        La communication est asynchrone. Dans un thread dédié, RhAL lit puis écrit
        les données sur le bus de communication, se synchronise avec le 
        thread de mouvement puis recommence. 
        Au sein du programme principal, les mouvements sont ainsi calculés dans un
        thread cadencé à la même fréquence que les communications bas niveau, 
        soit environs $100$~Hz.
        Les sources de la bibliothèque sont librement disponibles à l'adresse :
        \url{https://github.com/Rhoban/RhAL}.\\
    \item \textbf{Calibration du modèle de caméra : }
        Un autre sujet essentiel au succès de l'équipe dans la compétition RoboCup 
        concerne le lien entre la caméra et le modèle géométrique direct du robot.
        Une fois la balle ou la base des poteaux de but repérées dans l'image
        fournie par la caméra, il faut en estimer sa position cartésienne 
        relativement au robot.
        Positionner les objets détectés dans le monde est indispensable 
        à la localisation du robot sur le terrain.
        Sachant que ces objets sont posés sur le sol, connaissant les ouvertures
        angulaires et la pose de la caméra grâce au modèle, des calculs géométriques 
        simples estiment la position de l'objet par rapport au robot.
        Mais à cause des nombreuses imperfections mécaniques, cette estimation
        est souvent entachée pour un objet situé à un mètre de l'ordre d'une dizaine
        de centimètres d'erreur.
        
        Au travers de plusieurs compensations angulaires, le modèle géométrique direct
        est corrigé afin de ramener cette précision à un niveau plus acceptable : 
        de l'ordre d'un centimètre d'erreur à un mètre de distance.
        Plus précisément, les paramètres de correction sont des constantes angulaires
        rajoutées au niveau des transformations entre deux solides de l'arbre mécanique.
        Les données expérimentales sont enregistrées en manipulant manuellement
        la caméra du robot (les moteurs du cou étant relâchés).
        A chaque fois, un pixel précis de l'image est aligné avec un point sur
        le sol à distance connue. De plus, différentes postures du buste du robot
        sont explorées afin de lever certaines redondances des corrections géométriques.
        Ces paramètres de correction sont identifiés par une technique
        de maximisation de la log-vraisemblance.
        Cette méthode statistique permet de prendre explicitement en compte 
        le bruit de mesure dans des données d'apprentissage.
        L'optimisation est laissé à l'algorithme CMA-ES.\\
    \item \textbf{Génération de mouvements de tir par optimisation de la dynamique inverse : }
        Au vu de la difficulté et du temps nécessaire au réglage manuel de
        mouvements stables sur les robots, la piste d'une génération 
        automatique a été explorée à la section \ref{sec:motion_generation}.
        Ces travaux se concentrent en particulier sur la synthèse de
        mouvements de tir en boucle ouverte par optimisation.

        Au final, seuls des mouvements de tir en simple support et 
        de transition double support vers simple support se sont révélés
        stables sur le robot physique.
        Il n'a pas été possible d'atteindre la puissance du tir
        expert (en double support) tout en conservant la stabilité.
        Ceci en raison de la forte divergence entre le comportement idéal 
        et le comportement réel du robot.\\
    \item \textbf{Implémentation d'un simulateur physique pour Sigmaban : }
        Le but du simulateur est d'utiliser le modèle dynamique direct du
        robot pour prédire son comportement au cours du temps à partir des
        ordres envoyés aux servomoteurs.
        Les concepts sur lesquels se fonde l'implémentation du simulateur
        sont présentés à la section \ref{sec:model_dynamics}.
        Ces travaux ne sont pas encore aboutis et de nombreuses expérimentations
        restent nécessaires pour valider le simulateur ainsi que la méthode
        d'identification de ses paramètres utilisée.
        Les premiers résultats sont listés à la section \ref{sec:motion_results}.\\
    \item \textbf{Calibration du modèle de bruit de l'odométrie : }
        À l'occasion de l'édition 2017 du \textit{German Open} (une compétition
        locale organisée par la communauté RoboCup), 
        l'amélioration de l'apprentissage du modèle correctif de l'odométrie 
        (présenté section \ref{sec:odometry_cmaes}) a été commencé.
        La technique de maximisation de la log-vraisemblance est appliquée
        pour d'une part, 
        pour prendre en compte les sources de bruit affectant les observations expérimentales, 
        et d'autre part pour estimer un modèle de bruit du déplacement 
        proportionnel aux ordres envoyés à la marche.

        Pour chaque observation, l'évaluation de la prédiction 
        est répétée plusieurs fois en simulant les sources de bruit expérimentales.
        Ensuite, la distribution des prédictions possibles
        est interpolée par une gaussienne multivariée.
        Le critère statistique de la vraisemblance entre cette distribution 
        et les observations expérimentales est calculé.
        C'est ce critère qui est alors optimisé par CMA-ES.
        Ceci afin de rechercher les paramètres du modèle de correction, 
        du bruit de déplacement, ainsi que du bruit venant de l'expérimentation
        expliquant aux mieux les données.\\
    \item \textbf{Générateur de marche \textit{QuinticWalk} : }
        La section \ref{sec:walk_limitations} fait état des multiples limitations
        du générateur de marche \textit{IKWalk}.
        Pour l'édition 2017 de la compétition RoboCup, un nouveau générateur de marche
        en boucle ouverte \textit{QuinticWalk} a été développé afin de répondre
        à certaines de ses limites. Ses sources sont disponibles à l'adresse 
        \url{https://github.com/RhobanProject/Model/tree/master/QuinticWalk}.

        Le principe général du mouvement reste celui de la \textit{IKWalk} :
        boucle ouverte et mouvements dans l'espace cartésien géométriquement paramétrés. 
        Mais de multiples détails ont été modifiés.
        L'utilisation de splines de degré $5$ assure une continuité des accélérations 
        et donc des couples au cours du mouvement.
        Le calcul de la trajectoire du ZMP théorique (qui est continue) 
        est tracée sur la figure \ref{fig:walk_quintic_zmp}.
        En exprimant toutes les positions cartésiennes dans le repère du pied de support, 
        fixe par rapport au monde, la planification de la position des pas est rendu plus aisée.
        Cependant, cela entraine une discontinuité des splines au moment du changement 
        de pied de support qu'il est nécessaire de traiter.
        La continuité des positions cartésiennes cibles est assurée même en cas 
        de mise à jour des paramètres du mouvement au milieu d'un cycle de marche.
        La marche en phase de double support est implémentée et permet aux robots 
        de se déplacer lentement à basse fréquence (moins de un Hertz).
        
        Qualitativement, le mouvement de marche est bien plus stable, en particulier
        sur nos robot de plus grande taille. 
        La nécessité du processus de stabilisation utilisant les capteurs de pression
        pour la robustesse des pas chassés est réduite.
        Il serait très intéressant d'étudier l'effet de ce changement de générateur 
        de marche sur le bruit des déplacements et la précision de l'odométrie comme 
        discuté à la section \ref{sec:odometry_lwpr}.
        La principale évolution suivante de ce générateur de marche serait
        de remplacer les polynômes quintiques par des trajectoires 
        générée au niveau du jerk, proposées par 
        \cite{broquere2008soft}, \cite{broquere2011planification}.\\
\end{itemize}

Pour donner un ordre de grandeur des développements logiciels réalisés
au cours de cette thèse, les chiffres suivants peuvent être mentionnés.
Sans tenir compte des implémentations propre au programme principal 
de contrôle des robots, \textit{RhobanServer}\footnote{Collectivement, le développement 
de \textit{RhobanServeur} (et ses dépendances) par l'équipe 
dépasse aujourd'hui les \textbf{100000} lignes C++.},
j'ai personnellement écrit les nombres de lignes de code C++ suivants :
\begin{itemize}
    \item Bibliothèque \textit{RhIO} : \textbf{9276}
    \item Bibliothèque \textit{RhAL} : \textbf{6380}
    \item Simulateur symbolique avec le formaliste d'Euler-Lagrange : \textbf{8506}
    \item Reste de l'implémentation personnelle, 
        comprenant entre autres de nombreux tests et expérimentations, 
        la modélisation géométrique et dynamique des robots, 
        la simulation, les calculs odométriques et apprentissage, etc. : \textbf{64697}
\end{itemize}

