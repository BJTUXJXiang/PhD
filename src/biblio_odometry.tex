
Les références bibliographiques données dans ce chapitre ne sont pas 
dans la majorité des cas exhaustives, les domaines de recherche considérés 
ici étant à la fois divers et chacun très riches.
Souvent, les travaux cités tentent d'illustrer la multiplicité des 
méthodes employées ainsi que l'évolution des solutions apportées aux 
différents problèmes considérés.
Néanmoins, l'accent est mis sur les travaux dont les techniques 
mises en oeuvre, le contexte ou les plateformes robotiques 
se rapprochent des expérimentations présentées dans ce manuscrit.

Dans un premier temps, les travaux concernant l'odométrie proprioceptive
et prédictive sont rapportés. Avec plus en détail, l'estimation et la correction
de l'odométrie proprioceptive sur les robots à roues et humanoïdes, 
les méthodes concurrentes d'odométrie visuelle
puis les efforts de planification de déplacement en lien avec l'odométrie prédictive.
Ensuite la littérature entourant la locomotion bipède est abordée,
puis les principaux travaux ayant traités de la mise au point de mouvements
de tir sur les robots humanoïdes sont mentionnés.

\section{Odométrie\label{sec:biblio_odometry}}

La modélisation mathématique du déplacement des robots 
est aussi vielle que la robotique mobile.
L'estimation de la localisation et le contrôle des robots à roues 
sont abondamment étudiés depuis la seconde moitié du XXe siècle, \cite{muir_kinematic_1987}.
Les modèles cinématiques classiques se classent en quatre grands types :
\begin{itemize}
    \item Les robots à roues différentielles. 
        La vitesse des deux roues ou des deux groupes de roues, parallèles,
        sont contrôlés indépendamment.
        Dans cette catégorie se trouve également les robots 
        à \og dérapage \fg (\textit{skid-steering}), 
        (typiquement les robots à chenilles) se basant nécessairement sur 
        le glissement pour changer de direction.
    \item Les robots à roues orientables. Comme sur une voiture, deux roues reliées
        par un essieu (non indépendantes) sont orientables par rapport au plan sagittal du robot.
    \item Les robots omnidirectionnels.
        A l'aide d'au moins trois roues omnidirectionnelles bien positionnées, 
        la vitesse du robot peut prendre n'importe quelle direction sur le plan
        du sol en translation comme en rotation.
    \newline
\end{itemize}

Une différence fondamentale distingue les robots à roues
des robots à pattes et humanoïdes du point de vu 
de leur modèle cinématique.
Les robots humanoïdes sont sous-actués relativement à leurs 
déplacements planaires alors que les robots à roues ne le sont pas.
Aucun degré de liberté pris séparément ne permet directement au robot humanoïde 
de contrôler sa pose sur le sol.
Une coordination et un enchainement complexe de mouvements est nécessaire afin
de réellement faire marcher le robot selon un vecteur donné.
A contrario, le contrôle en vitesse ou de la direction des roues permet de manière simple
au robot de contrôler son déplacement\footnote{à l'exception des robots 
à roues omnidirectionnels, dans une bien moindre mesure}.

\subsection{Calibration de l'odométrie}

Une des premières contribution marquante à la correction de l'odométrie 
sur les robots à roues différentielles est proposée en 1996 
par \cite{borenstein_measurement_1996}.
Leur objectif est ici d'identifier les imperfections mécaniques
du robot à roues affectant son déplacement : diamètres des deux 
roues légèrement différents, défaut de parallélisme et montage mécanique
des roues l'une par rapport à l'autre.
Le modèle de déplacement mathématique est simple et sous forme linéaire.
Les auteurs dérivent alors analytiquement les conséquences géométriques 
sur la trajectoire de ces erreurs mécaniques.
Un de leur principal apport est la définition d'une procédure de test 
standard (\textit{benchmark} UMBmark) évaluant la qualité de l'odométrie 
sur un robot à roues.
Les trajectoires de références en formes de carrés à parcourir 
dans les deux sens mettent en exergue les défauts du modèle.
Puis en mesurant la trajectoire géométrique réelle du robot, les auteurs
en déduisent directement par le calcul la valeur des paramètres du modèle
de déplacement décrivant des défauts mécaniques.

\cite{roy_online_1999} proposent dès 1999 la calibration d'un modèle 
de déplacement utilisant la méthode de maximisation de la log-vraisemblance 
(\textit{log-likelihood maximization}) sur les robots à roues.
Ils s'appuient sur un capteur de distance laser monté sur le robot 
mobile permettant au robot d'observer (dans de bonnes conditions) 
son déplacement et ainsi de calibrer son odométrie pendant son fonctionnement 
sans intervention humaine.
Deux paramètres représentant la dérive proportionnelle au temps de 
la translation et de la rotation du robot sont analytiquement optimisés 
de sorte que l'odométrie estimée soit la plus vraisemblable sachant le 
déplacement observé.
Comme l'intégration en forme close de l'odométrie n'est pas mathématiquement
praticable, les auteurs ne considèrent l'odométrie qu'entre deux mesures 
successives du capteur laser.
À noter cependant que l'estimation du bruit de déplacement 
n'est pas traité ici.
L'article annonce une réduction moyenne de l'erreur de 83\%, passant
par exemple de $18$~m à $3.08$~m d'erreur en position après avoir parcouru
$269$~m.

Plus théorique, \cite{kelly_fast_2004} développe une méthode générique 
statistique pour l'identification des paramètres du modèle de déplacement.
L'auteur propose également d'estimer la covariance de ces paramètres.
Le modèle cinématique, à priori non linéaire est linéarisé et différentié 
numériquement au premier ordre selon chacun de ses paramètres.
Le robot est contrôlé pour suivre un ensemble de trajectoires prédéfinies. 
Pour limiter l'effort humain, la position et l'orientation réelle du
robot n'est mesurée qu'une seule fois à la fin de la trajectoire. 
Il n'y a ainsi pas besoin d'un système de mesure externe.

Les robots à chenilles (différentielles) sont particulièrement 
difficiles à contrôler. 
Pour pouvoir tourner, les robots à chenilles se reposent nécessairement 
sur un certain glissement avec le sol et dépendant de la surface.
Afin de simplifier leur modèle cinématique, \cite{martinez_approximating_2005}
mettent en œuvre un algorithme d'optimisation génétique en boite noire.
Ils identifient les paramètres d'un modèle classique simple d'un robot à roues 
différentielles capturant le comportement de leur robot à chenilles.
Le mouvement du robot réel est découpé en courtes séquences et 
ses poses initiales et finales sont mesurées par un capteur externe.
Comme l'intégration du déplacement ne s'exprime pas de manière simple, 
un algorithme en boite noire ne demandant pas la connaissance du gradient
est employé.
La distance entre le déplacement observé et l'odométrie estimé 
selon le modèle est alors minimisée.
Le bruit de mesure ou de déplacement n'est ici pas pris en compte.

Sur de petites plateformes, \cite{antonelli_calibration_2005}
transforment analytiquement le modèle cinématique classique des robots 
à roues différentielles afin de rendre le problème d'identification linéaire
en ses paramètres.
Une matrice de quatre paramètres est utilisée pour représenter
les trois paramètres du modèle cinématique classique sous forme linéaire.
Sous cette forme, l'odométrie des robots peut être analytiquement intégrée 
et la méthode des moindres carrées est alors appliquée.
Les poses initiales et finales des robots sont capturées par un système 
de détection visuelle de marqueurs placés au dessus des robots.
La forte originalité de leurs travaux tient dans l'analyse théorique 
du problème permise par leur mise sous forme simple.
Les auteurs insistent sur le fait que les trajectoires en boucles fermées et
les trajectoires dont les rotations se compensent induisent un mauvais 
conditionnement de la régression linéaire.
Une bonne trajectoire d'apprentissage ne revient donc pas à son point 
de départ et l'orientation du robot évolue de manière monotone.
Une trajectoire en forme de \og L \fg est ainsi préconisée.
Les résultats expérimentaux précisent à la fois la moyenne, le maximum 
et la variance de l'erreur en position de l'odométrie et le bruit 
de quantification est également discuté.

\cite{angelova_learning_2007}, proposent quant à eux d'utiliser une méthode
d'apprentissage supervisée (ou régression) non paramétrique sur un robot 
à roues de type rover d'exploration planétaire.
Leur but est de prédire à partir d'informations visuelles le pourcentage 
de glissement des zones à l'avant du robot, affectant fortement son odométrie.
Les auteurs emploient l'algorithme LWPR (\textit{Locally weighted projection regression}) 
et le compare aux réseaux de neurones avec une préférence pour LWPR.

Toujours dans le but d'identifier les paramètres du modèle de déplacement
de robots à roues différentielles, \cite{ivanjko_simple_2007} appliquent un algorithme
d'optimisation non linéaire hors ligne.
La méthode d'optimisation non linéaire de Gauss-Newton
ne demande pas de fournir le gradient de la fonction de récompense à minimiser.
L'optimisation minimise la distance entre la position cartésienne estimée par l'odométrie
et le déplacement réel mesuré manuellement après une trajectoire du robot.
Les auteurs comparent alors la qualité de l'apprentissage et la précision 
obtenue de l'odométrie selon deux modèles. Le premier a seulement deux paramètres
et le deuxième en a trois.
Comme très peu de trajectoires d'apprentissages sont ici récoltées et utilisées, 
il s'avère que le modèle à seulement deux paramètres s'avère plus performant.

Enfin toujours sur les robots à roues, \cite{censi_simultaneous_2013} remarquent
que de la même manière que le problème SLAM (localisation et cartographie), 
le problème de la calibration de l'odométrie et de la calibration d'un capteur externe permettant 
cette calibration sont étroitement liées.
Les auteurs proposent dans cet article une méthode permettant à l'aide de la technique
de maximisation de la log-vraisemblance de calibrer à la fois l'odométrie proprioceptive 
ainsi que le modèle de capteurs extéroceptifs du type télémètre laser mesurant le déplacement 
externe du robot.\\

Avant l'introduction des robots humanoïdes NAO en 2008 dans 
la ligue des plateformes standards (\textit{standard platform league}) 
de la RoboCup, la compétition se basait sur les petits robots quadrupèdes AIBO
(de 1999 à 2008).
Il est important de saluer l'impact précurseur des travaux de recherche
réalisés durant cette période.
Une grande partie de la littérature liée
aux robots humanoïdes jouant au football (locomotion, odométrie, 
localisation, perception, tir, calibrations et apprentissages diverses) 
est en réalité inspirée par des travaux analogues expérimentés 
en premier lieu sur les plateformes AIBO\footnote{\url{http://www.sony-aibo.com/}}.

Par exemple sur le thème de l'odométrie, \cite{hoffmann_exploiting_2005} mettent 
en évidence les difficultés de l'estimation de l'odométrie proprioceptive 
sur un robot à pattes (complexité de la topologie mécanique). 
Notamment en raison des collisions avec les autres robots qui entrainent alors
un glissement important.
Un modèle linéaire du bruit du déplacement du robot est ajusté manuellement.
De plus, la détection des collisions avec les autres robots leur permet
de construire un critère de \og fiabilité \fg de l'odométrie.
Cet critère est utilisé pour améliorer la localisation du robot sur le terrain 
au travers d'un filtre particulaire.

On peut également citer \cite{he_model-based_2007} qui mettent en place
un algorithme d'optimisation génétique afin d'identifier les paramètres
d'un modèle géométrique et physique de leurs robots AIBO.
L'optimisation se base sur l'enregistrement d'une douzaine de séquences
de marche où le déplacement réel sur le robot est mesuré manuellement.
Grâce aux capteurs binaires de contact sous les pieds du robot AIBO et 
du modèle optimisé, ils améliorent l'estimation de l'odométrie proprioceptive.
Le bruit du déplacement n'étant pas alors considéré.

\subsection{Odométrie visuelle}

Devant la difficulté à réellement mesurer le déplacement
d'un robot mobile à l'aide de capteurs uniquement \og proprioceptifs \fg
(n'observant que des grandeurs internes au robot), 
une autre piste de recherche a été explorée.
L'observation de grandeurs externes a l'intérêt de pourvoir estimer 
le déplacement relatif du robot de manière \og absolue \fg ; ou plus
exactement de ne pas avoir à considérer la façon dont le robot interagit
avec le sol (modèle cinématique, glissements).

Par exemple, \cite{czarnetzki_odometry_2010} ont intégrés au niveau des pieds 
d'un petit robot humanoïde NAO des capteurs optiques similaires 
à ceux des souris optiques. 
Ils permettent de mesurer directement le déplacement des pieds par 
rapport au sol et d'accéder précisément au glissement du pied de support. 
Néanmoins, la sensibilité de ces capteurs est très dépendante de leur hauteur
par rapport au sol. Ils sont difficilement exploitables pour 
le déplacement du pied en vol d'un humanoïde.

Les capteurs lasers de types LIDAR (\textit{Light Detection And Ranging}) 
sont très utilisés pour leur précision et leur relative simplicité d'analyse. 
Typiquement, \cite{hornung_humanoid_2010} utilisent un LIDAR monté 
sur la tête d'un robot humanoïde NAO afin d'estimer son odométrie ainsi que
sa localisation absolue dans un environnement 3D connu.

Malheureusement, ces instruments sont mécaniquement fragiles.
La moindre chute du robot NAO peut gravement endommager les parties 
mobiles du LIDAR monté sur sa tête.
De plus, bien qu'en constante baisse depuis leur production en masse dans
les robots aspirateurs, leurs prix restent élevés en comparaison
du coût des petits robots humanoïdes.

Les LIDAR sont surtout très employés sur les robots mobiles à roues par
la communauté SLAM (\textit{simultaneous localization and mapping}).
Ce domaine de recherche s'intéresse au problème de localisation et
cartographie simultanée. 
Le but étant à la fois, de construire une carte d'un environnement 
inconnu et de s'y positionner le plus précisément possible afin de
pouvoir y naviguer.
Ce pan de la recherche à l'intersection de la robotique et 
du traitement d'image et du signal est très riche, très actif et 
en constante progression depuis plus de vingt ans 
(résumé par \cite{fuentes-pacheco_visual_2015}).

Le domaine du SLAM recourt également beaucoup aux
caméras, monoculaires ou binoculaires. 
Très riches en informations, mécaniquement plus robuste que 
le LIDAR et à relativement bas coût, elles sont également bien 
plus difficiles à analyser.
Les méthodes de SLAM visuelles sont particulièrement complexes 
à mettre en oeuvre en extérieur et dans des environnements 
très dynamiques ou pauvres en caractéristiques visuelles saillantes.
De plus, les limitations liées au temps de calcul et à la latence 
compliquent leur emploi dans des applications embarquées 
typiques de la robotique.

Néanmoins, des méthodes d'odométrie visuelle ont été développées depuis 
2004 et sont directement dérivées des techniques de SLAM visuelles. 
Leur but étant d'estimer l'odométrie de la camera se déplaçant dans
le monde en ne se basant que sur l'exploitation des images.
\cite{nister_visual_2004} est le premier article introduisant 
ces méthodes. Ils n'utilisent aucune information proprioceptive.
Uniquement les images de la camera (ou des caméras).
Des caractéristiques visuelles sont repérées puis appairées dans les images
successives afin d'en déduire la différence de pose de la caméra.
Ces déplacements relatifs sont ensuite intégrés au cours du temps.
L'article détaille également les nombreuses optimisations techniques nécessaires
afin d'obtenir une fréquence de traitement acceptable de l'ordre de $13$~Hz
pour une résolution de 720x240 mais non encore embarquable.

Depuis, la recherche à perfectionner les techniques afin d'en réduire
les temps de calculs et de les rendre plus robuste.
Par exemple, \cite{weiss_monocular-slambased_2011} 
estime l'odométrie visuelle à bord d'un petit hélicoptère télécommandé.
La méthode SVO (\textit{Semi-direct Visual Odometry}) introduite par
\cite{forster_svo:_2014} et ses développements sont en 2017 considérés 
comme état de l'art et fonctionnent sur de petits quadrirotors volants.
Plus généralement, la recherche sur l'odométrie visuelle s'est beaucoup
orientée vers les applications concernant les voitures autonomes ainsi que 
sur les drones afin d'en améliorer le contrôle en intérieur et en 
complément (ou absence) du GPS.

Appliquées aux robots humanoïdes, les méthodes d'odométrie visuelle
font face à encore d'autres difficultés.
Contrairement aux drones ou aux voitures, la caméra 
(principalement fixé au niveau de la tête) des robots humanoïdes 
est soumis pendant la marche à un important mouvement d'oscillation 
latérale ainsi qu'à des chocs provoqués par l'atterrissage du pied en vol
à chaque pas. 
Ceci induit un flou des images capturées rendant délicat la détection 
et l'appariement des caractéristiques visuelles.

La première application de l'odométrie visuelle sur un robot humanoïde
date de 2006 par \cite{stasse_real-time_2006} sur un grand robot HPR-2.
Afin de répondre aux problématiques énumérées ci-dessus et d'améliorer la
robustesse, les auteurs s'appuient en plus de la caméra sur le modèle de
déplacement du robot.
Plus en détail, le générateur de marche fournit une première
estimation du déplacement espéré du robot qui est ensuite mélangée à l'odométrie
visuelle au travers d'un filtre de Kalman étendu (le modèle 
de déplacement étant non linéaire).
L'information du déplacement prévu est utilisé afin d'aider l'analyse d'image
à atteindre une vitesse de traitement en temps réel à bord du robot.

Il est néanmoins nécessaire de noter que contrairement aux petits robots
très imparfait dans leur mécanique et dans leur contrôle, la tête du robot HRP-2
n'est pas soumis à des chocs très importants.
Sur un petit robot humanoïde, \cite{pretto_visual_2009} proposent
une nouvelle détection de caractéristiques visuelles plus robuste au flou
typique de la marche des robots et améliore un algorithme classique
d'odométrie visuelle.

Sur un robot plus grand soumis à de forts chocs à cause de sa marche 
utilisant les talons, \cite{ahn_-board_2012} réduisent grandement
la dérive de l'odométrie en se basant en plus de la caméra sur le modèle 
géométrique direct du robot.
La lecture des capteurs proprioceptifs permet au travers du modèle
du robot de calculer une estimation de son déplacement relatif à chaque pas.
Un filtre de Kalman étendu est alors appliqué afin de mélanger
l'estimation de l'odométrie proprioceptive et l'odométrie visuelle.
L'intérêt est double. D'une part, l'odométrie visuelle corrige à 
\og basse fréquence \fg l'estimation du déplacement à partir 
d'informations externes.
D'autre part, l'intégration des encodeurs des moteurs est
calculé à \og haute fréquence \fg et complète l'état du robot entre
deux prises d'images de la caméra.

Enfin, les travaux de \cite{oriolo_humanoid_2016} aboutissent à une dérive
de l'odométrie particulièrement faible sur le petit robot humanoïde NAO.
Les auteurs annoncent une dérive de la position réelle du robot
de l'ordre de seulement $10$~cm après $1.2$~m de déplacement.
À noter néanmoins que l'analyse d'image est ici effectué de manière 
déportée sur un ordinateur fixe plus puissant.
Plus précisément, et comme précédemment, le modèle géométrique direct
proprioceptif est couplé à l'odométrie visuelle à l'aide d'un filtre
de Kalman étendu. 
De plus, les auteurs tirent parti des capteurs de pression
(capteurs FSR, \textit{Force Sensitive Resistor})
sous les pieds du robot NAO pour améliorer l'estimation du déplacement 
avec une meilleure détection du pied de support courant.
L'article détaille l'intérêt de l'amélioration de la précision de l'odométrie 
dans plusieurs expériences de localisation et d'asservissement de la navigation.
Finalement, les auteurs insistent sur la difficulté de configuration
des matrices de covariances du filtre de Kalman et sur les nécessaires 
calibration et phase d'initialisation de la caméra.

\subsection{Planification de trajectoires}

Quand l'odométrie des robots leur permet de percevoir leurs
déplacements et leur localisation, le modèle cinématique de 
déplacement est un élément clef dans le contrôle et la planification
de la trajectoire de navigation.
La mise au point d'une politique de contrôle du déplacement du robot
est discutée et réalisée à la section \ref{sec:odometry_cmaes}.
Comme pour l'odométrie, ce sujet est depuis le début
de la robotique mobile un domaine de recherche très étudié, 
en premier lieu sur les robots à roues.
Le modèle de déplacement théorique simple des robots à roues,
le plus souvent exprimé sous forme matricielle linéaire
(\cite{muir_kinematic_1987}) autorise de multiples formes
de planifications, analytiques ou non.
Les principaux biais affectant la modélisation du déplacement
des robots à roues résident dans leurs défauts de montages mécaniques 
mais surtout dans les glissements intervenant entre les roues et 
le sol.

Afin de prendre en compte ces effets dans le déplacement du robot,
il est nécessaire de calibrer ou d'identifier un modèle de déplacement 
capturant ces phénomènes.
Le point important à noter est que ces modèles peuvent être
construis à partir des mêmes données que celles utilisées pour
l'apprentissage de l'odométrie.
\cite{howard_model-predictive_2014} évoquent différents modèles
de déplacements pour les robots à roues et synthétisent les multiples
techniques de planifications de trajectoire utilisant ces modèles
prédictifs.

Par exemple toujours sur un robot à roues de type 
\og rover martien \fg, \cite{howard_optimal_2007} proposent une
méthode de planification de trajectoire optimale sur un terrain 
très accidenté.
Le chemin calculé par un algorithme d'optimisation itératif prend
en compte les glissements des roues en les simulant au travers d'un modèle
linéaire simple (l'identification du modèle n'est pas traité).
Les auteurs insistent sur l'importance de tenir compte des glissements
dans la commande en boucle ouverte (pré-compensation) car l'asservissement 
réactif ne peut pas seul corriger la trajectoire.

Chez les robots humanoïdes, le modèle de déplacement passe
par la prédiction et la planification de la positions des pieds 
ou pas au sol.
Une des premières réalisation de planification de pas 
date de 2005 par \cite{chestnutt_footstep_2005} sur le grand 
robot humanoïde ASIMO.
Il s'agit de planification de pas afin d'éviter des surfaces 
interdites planaires. Le problème des glissements n'est pas abordé
et les positions des pas prévus sont supposés exactes ; ce qui est
généralement le cas sur de ce type de robots très précis.
La planification des pas au long terme est très couteuse en temps
de calcul. Les auteurs sont ainsi amener à faire de nombreuses approximations 
et discrétisations des pas possibles afin de pouvoir employer 
l'algorithme classique de recherche de chemin A*.

Également sur un grand robot HRP-2, \cite{perrin_fast_2012} présentent
une approche basée cette fois sur l'algorithme RTT pour la planification 
rapide de pas évitant un ensemble d'obstacles 3D sur le sol.
Une étude de la forme du volume des actions réalisables est conduite afin
de pouvoir approximer cet espace et ainsi réduire et accélérer l'exploration
de l'espace d'action.

Dans un contexte RoboCup très proche du notre, \cite{schmitz_real-time_2012}
planifient les pas d'un robot humanoïde de taille intermédiaire afin 
d'approcher et d'orienter un des pieds du robot devant une balle dans
une bonne position de tir.
Les auteurs ne considèrent aucun obstacle mais prennent en compte la dynamique
du robot en imposant des limites en vitesse et accélération.
La planification utilise l'algorithme A* avec discrétisation des actions 
en accélération. 
Néanmoins, le temps de calcul étant prohibitif, une politique
d'action n'encodant que le prochain pas est construite après une
longue phase de pré-calcul.
Plusieurs techniques de représentations de la politique sont alors 
comparées. Elles permettent d'interroger en temps réel 
sur le robot physique le résultat de la planification.

Cette planification fait suite aux travaux \cite{schmitz_learning_2010} par
les même auteurs. Ils mettent alors en place l'apprentissage d'un
modèle linéaire du déplacement du robot en se basant sur un système de capture de mouvement.

Enfin, \cite{hornung_search-based_2013} mettent en oeuvre et évaluent 
la méthode R* (\textit{Randomized A*}) sur le robot humanoïde NAO
et réussissent à obtenir une planification des pas en temps réel.
La planification des pas permet l'évitement et l'enjambement d'obstacles 
3D sur le sol.
Pour cela, la méthode R* combine à la fois l'exploration par échantillonnage
de l'algorithme RTT mais au lieu d'une exploration aléatoire, la recherche
est guidée par la minimisation d'un coût de la même manière que l'algorithme A*. 
La méthode ARA* (\textit{Anytime Repairing A*}) est également comparée. 
Ces méthodes ont pour principale caractéristique de retourner une solution
approchée (\textit{anytime}) du problème de planification même 
lorsqu'elles sont interrompues au bout d'un temps arbitraire.\\

Ainsi, les méthodes appliquées à la planification du déplacement du robot
reposent sur deux grandes familles : 
les variantes des algorithme A* et RTT (\textit{rapidly exploring random tree}).
Ces deux types de méthodes présupposent que le déplacement du robot
est déterministe. Aucun bruit n'est considéré.
A contrario, \cite{ApproachICAPS2017} (voir section \ref{sec:odometry_cmaes}) proposent 
l'utilisation du formalisme des processus de décisions Markovien (MDP)
pour prendre en compte le bruit de déplacement, très présent sur nos robots.

\section{Locomotion bipède\label{sec:biblio_walk}}

Le sujet de la marche humanoïde
est un domaine de recherche très particulier.
D'une part, l'expérience quotidienne de la locomotion
humaine rend ce sujet immédiatement compréhensible
au grand public. 
L'exemple des capacités motrices de l'humain, de sa réactivité et
de son adaptabilité, rappellent tous les jours 
l'ambition de l'objectif à atteindre.
De plus, la marche des robots est intimement liée au rêve
collectif du robot humanoïde anthropomorphe.

À la fois, il est relativement facile de \og bricoler \fg
un mouvement faisant faiblement bouger un robot humanoïde
en glissant les pieds sur un sol parfaitement plat.
Autant la démonstration d'un mouvement de marche en boucle fermée,
rapide, robuste à la déstabilisation de perturbations extérieures
et s'adaptant à différents terrains accidentés (sable, neige, montagne, ...) 
reste aujourd'hui un problème ouvert\footnote{Il faut cependant saluer les avancées 
exceptionnelles du robot humanoïde \textit{Atlas} de l'entreprise \textit{Boston Dybamics}, 
et du robot bipède \textit{Cassie} de \textit{Agility Robotics} (successeur de \textit{ATRIAS}).}.
Plus généralement, l'intelligence sensorimotrice des robots est encore loin 
d'égaler celle observée chez les animaux.
Il est toutefois important de noter que la motricité des robots est avant tout 
contrainte par les capacités des actuateurs ; loin de posséder le ratio poids-puissance
et l'efficacité énergétique de leurs équivalents biologiques.

La littérature de ce sujet est particulièrement abondante et diverse.
De nombreux domaines de recherche ont en effet abordé le problème
et de nombreuses méthodes ont été expérimentées sur 
la locomotion et de la stabilité bipède.
Les quelques travaux mentionnés ci-dessous, classiques ou non,
ont contribué à nourrir les réflexions de cette thèse.
Plus précisément, j'ai mis au point et implémenté pour l'équipe Rhoban
les générateurs de marche omnidirectionnels \textit{IKWalk} (voir section \ref{sec:walk})
et \textit{QuinticWalk} (section \ref{sec:other_works}).
Ces développements sur les petits robots humanoïdes de l'équipe
ont été inspirés par les travaux suivants :\\

Pour commencer, il faut citer \cite{vukobratovic_zero-moment_2004},
proposant le concept classique du ZMP (\textit{Zero Moment Point}) 
(voir section \ref{sec:zmp}).
Cette notion de dynamique définie un critère suffisant mais non nécessaire 
de stabilité pour la marche des robots à pieds plats, sur sols plats
(depuis étendu aux sols non plats).

Les travaux de Shuuji Kajita, \cite{kajita_3d_2001} et \cite{kajita_biped_2003},
représentent la méthode \og traditionnelle \fg de marche 
des robots humanoïdes.
La dynamique du robot est modélisée par la linéarisation du pendule inversé simple
(\textit{LIPM}, \textit{Linear Inverted Pendulum Model}).
Toute la masse du robot est placée à hauteur constante au niveau de 
son centre de masse (\textit{COM}) et attaché au sol à la position du
pied de support par une tige sans masse (de taille variable).
En choisissant arbitrairement la pose des pas du robot sur le sol,
le problème revient à déterminer la trajectoire du COM afin que
ses accélérations maintiennent le ZMP au niveau du pied de support.
Grâce à la linéarisation, la dynamique du robot ainsi 
que le critère du ZMP s'expriment sous forme matricielle linéaire
et le problème est alors résolu en tant que problème d'asservissement.
La méthode du précontrôle (\textit{preview control}) basée
sur la théorie du régulateur linéaire quadratique 
(\textit{LQR}, \textit{Linear Quadratic Regulator}) est appliquée.

Sur un robot humanoïde de taille intermédiaire, 
\cite{behnke_online_2006} propose un des premiers mouvements 
de marche omnidirectionnelle.
La spécificité de cette approche est qu'elle ne se base pas 
sur le critère du ZMP ni sur aucune modélisation dynamique du robot.
Elle n'est pas réactive (boucle ouverte) en dehors de l'asservissement en
position des servomoteurs et ne demande qu'une puissance de calculs très réduite.
Le mouvement d'oscillation géométrique du robot est directement paramétré 
dans l'espace articulaire.
La stabilité et les performances du mouvement reposent sur le réglage
manuel des différents paramètres du mouvement.
Le contrôle du mouvement omnidirectionnel se fait à l'aide 
de trois paramètres : vitesse d'avance, latérale et de rotation.
Cette approche est très populaire dans la communauté des robots de la
ligue humanoïdes de la RoboCup par la simplicité des concepts mis en jeu.
Le mouvement de marche \textit{IKWalk} utilisé sur nos robots suit
une méthode similaire mais dans l'espace cartésien
(voir la présentation du générateur à la section \ref{sec:walk}).

Les différents travaux portés par Pierre-Brice Wieber, 
\cite{wieber_trajectory_2006}, \cite{diedam_online_2008}
sont fondés sur l'application du contrôle prédictif 
(\textit{Model Predictive Control}, \textit{MPC}) à la locomotion bipède.
L'idée repose sur d'une part, la prédiction des états 
futures du système à partir d'une séquence d'actions possibles à horizon fini.
Et d'autre part, à optimiser cette séquence d'actions afin que l'état futur 
du système rejoigne au plus vite l'état désiré.
À chaque itération de la boucle de contrôle du robot, 
la séquence d'ordres futures optimale est optimisée et 
seule la première action est appliquée.
L'opération est alors continuellement répétée et doit
donc être rapide en temps de calcul.
C'est grâce à la linéarisation du modèle du pendule inversé
que le comportement du robot se formalise comme un système dynamique linéaire.
Le contrôle du ZMP peut ainsi se réduire à la résolution d'un problème 
d'optimisation quadratique (optimisation sous contraintes).
La difficulté consiste en effet en la gestion des différentes
contraintes (inégalités), par exemple les limites d'accessibilité 
imposées par le modèle géométrique inverse.
Le deuxième article traite de l'utilisation du positionnement 
des pieds comme dimensions d'actions supplémentaires
pour conserver l'équilibre bipède.
Cependant, ces travaux se reposent fortement sur une modélisation
et un contrôle précis du robot.
Ces méthodes sont en pratique difficilement applicables à nos robots imparfaits.

\cite{graf_closed-loop_2010} et \cite{graf_center_2011} adaptent les travaux 
de Kajita sur le petit robot humanoïde NAO.
Une marche omnidirectionnelle sans phase de double support est proposée.
Le mouvement est ici réactif à la trajectoire réelle du centre de masse.
La pose du buste est estimée au travers des capteurs (encodeurs de positions et IMU)
et du modèle géométrique direct.
De plus, des filtres de Kalman sont utilisés pour filtrer la position mais 
aussi la vitesse du centre de masse du robot.
Grâce aux filtres, l'état du pendule inverse à tout moment est reconstruit ; 
le filtrage permet de résoudre le problème de différentiation numérique 
de la position mesurée bruitée du buste.
Quand une perturbation de la trajectoire du COM est détectée, une nouvelle
trajectoire du COM est générée et les pas futurs du robot sont modifiés.
Enfin, la latence entre l'écriture des ordres moteurs et les lectures
des encodeurs est également gérée au travers du filtre.

Les travaux de Marcell Missura, \cite{missura_lateral_2011}, \cite{missura_omnidirectional_2013},
\cite{missura_learning_2014} et \cite{missura_analytic_2016} présentent une méthode
appelée \og pas de capture \fg \textit{capture step framework} pour la stabilisation
réactive de la marche bipède sur un robot humanoïde de taille intermédiaire.
Cette technique est l'extension du \og point de capture \fg 
(\textit{capture point}) par \cite{pratt_capture_2006}.
Il s'agit d'un point sur le sol tel que le robot peut placer son ZMP sur ce point
pour stopper son mouvement en un pas.
En développant cette approche, l'auteur propose une méthode contrôlant
la trajectoire du centre de masse en agissant en permanence sur 
la position et la durée des pas du robot.
Le mouvement de marche est en grande partie impulsé par des oscillations 
en boucle ouverte.
La détection des perturbations de la trajectoire du COM
passe par le calcul de l'énergie orbitale du pendule, sensée être constante.
Les équations en formes closes de la trajectoire du COM permettent de prédire
ses futures positions théoriques.
Ces équations sont utilisées pour calculer le moment et la position du prochain
pas pour ramener la trajectoire du COM à une trajectoire nominale.
Cette méthode est de plus peu couteuse en temps de calcul.
Le bruit de l'IMU est pris en compte et les pas du robot ne sont pas
modifiés lorsque l'incertitude sur la vitesse de COM est trop importante.
Enfin, la dernière étude de l'auteur étend la méthode du pas de capture par l'ajout
d'une étape d'apprentissage.
L'intérêt de cette approche est qu'elle constitue un exemple
de mouvement de marche réactive s'appliquant avec succès sur 
des robots de taille intermédiaire et imparfaits, très similaires à nos propres robots.

Sur un robot NAO simulé, \cite{hugel_walking_2012} proposent également
une marche se basant sur le modèle linéaire du pendule inversé
et le critère ZMP.
Plus spécifiquement, les auteurs font état dans cet article de la phase 
de démarrage de la marche ainsi que du recollement de la trajectoire 
du centre de masse entre deux primitives motrices.
Une approche ressemblante est employée au sein du nouveau mouvement de
marche en boucle ouverte, la \textit{QuinticWalk} développée en 
2017 et mentionnée à la section \ref{sec:other_works}.

Pour finir toujours sur le robot NAO, une approche très intéressante est
développée par \cite{seekircher_closed-loop_2016} en se basant
également sur une marche par pendule inversé et ZMP. 
Tout d'abord, leur modèle de pendule inverse est augmenté 
de termes de compensations et de décalages afin de prendre en compte
les imperfections systématiques des robots.
Ensuite, pendant son fonctionnement, la dynamique du robot
est continuellement observée. 
Un algorithme d'optimisation en boite noire (CMA-ES),
optimise les paramètres du modèle du pendule afin d'améliorer 
la prédiction du comportement récent du robot.
L'ajustement des paramètres est ainsi recommencé à intervalles réguliers
au cours du fonctionnement du robot.
Étant donné que le contrôle de la marche est basé sur ce modèle, 
la stabilité du mouvement s'en trouve améliorée et s'adapte 
en quelques minutes à une nouvelle surface du sol (herbe artificielle, moquette).

\section{Synthèse, optimisation et simulation de mouvements\label{sec:biblio_motion}}

Ci-dessous, quelques travaux en relation avec la synthèse de mouvements 
(voir section \ref{sec:motion_generation}) sont présentés.
Tout d'abord concernant la mise au point d'un mouvement de tir notamment 
sur des grands robots \og précis \fg et sur petits robots imparfaits. 
Puis quelques travaux s'étant intéressés au problème du transfert de mouvements du simulateur
à la réalité sont mentionnés.\\

Sur le petit robot quadrupède AIBO, \cite{hausknecht_learning_2011}
mettent en place une installation expérimentale permettant de répéter
un très grand nombre de fois un mouvement de tir sur le robot réel.
En évaluant la distance de tir sur le robot, un algorithme d'optimisation 
par descente de gradient optimise les $10$ paramètres d'un mouvement
représenté par des splines polynomiales.\\

Sur le grand robot HRP-2, \cite{miossec_development_2006} proposent 
l'un des tout premiers travaux sur la génération d'un mouvement de tir
(en simple support).
Pour commencer, une phase d'identification des paramètres inertiels
du robot améliore la qualité de la modélisation dynamique.
Ensuite, un algorithme d'optimisation par descente de gradient non linéaire
est utilisé pour minimiser, sous contraintes, la somme des couples
d'un mouvement de tir représenté par des B-splines.
Ces contraintes incluent entre autre la position du ZMP.
Le problème du double support support n'est pas traité.

Sur le robot HRP-2 (\cite{lengagne_generation_2010}), puis
sur le petit robot humanoïde HOAP-3 (\cite{lengagne_planning_2011}), 
les auteurs proposent un perfectionnement de la technique de génération 
de mouvements par optimisation de la dynamique inverse. 
Ils réduisent entre autres les temps de calculs et améliorent la gestion
du respect des contraintes.

\cite{xu_adaptive_2010} découplent sur le robot NAO le mouvement 
de tir du problème de la stabilisation.
Au niveau de la jambe de tir, un mouvement est déterminé géométriquement
pour frapper la balle à une position et dans une direction données.
Le problème de la stabilité est géré au niveau de la jambe de support par
un contrôle proportionnel sur le centre de masse.
La trajectoire du pied en vol est représentée par des splines de Bézier
et découpée de manière experte en différentes \og phases \fg.
La zone de stabilité du tir est alors étudiée.
Le problème de cette approche est que le mouvement prévu de la jambe
de tir n'est pas utilisé dans la stabilisation purement réactive.
Une approche très semblable est suivit par \cite{ferreira_development_2012}
sur le robot NAO en pure simulation pour la mise au point d'un tir omnidirectionnel.

Une approche originale est expérimentée sur le petit robot humanoïde
DARwIn-OP par \cite{yi_improved_2013}. Le problème du tir est relié à celui
de la marche. La même théorie du pendule inverse linéaire et du contrôle
du ZMP (\textit{preview control}) est appliquée à la stabilisation du tir. 
Le problème de l'enchainement de la marche et du tir est en particulier traité.

En simulation avec le robot NAO, \cite{jouandeau_optimization_2014} proposent
la paramétrisation d'un mouvement de tir puis l'optimisent
afin d'en maximiser la portée. 
Une méthode d'optimisation sans gradient est utilisée.
Une des spécificités de ces travaux est la
proposition d'utiliser la rotation du torse du robot pour accompagner 
la jambe de tir et maximiser la puissance.
Le problème de la stabilisation n'est pas considéré.

Sur le robot NAO réel, \cite{wenk_online_2014} se basent également
sur le modèle du pendule inversé linéaire pour stabiliser le mouvement tir.
Grâce à la proposition d'une méthode d'estimation du ZMP avec un filtre de Kalman,
ce dernier est asservit sous le pied de support du robot.
Deux méthodes de contrôle sont alors comparées.
Toutefois, comme la trajectoire du pied en vol est complètement décorrélée du problème
de stabilisation, l'information de sa trajectoire (connue à l'avance) n'est pas utilisée.
Le mouvement de la jambe de tir est ainsi perçu comme une perturbation extérieure
et il n'y a donc pas de possibilité de pré-compensation.

Toujours dans le contexte de la ligue humanoïde standard NAO de la RoboCup, 
\cite{bockmann_kick_2016} utilisent des DMP (\textit{Dynamic Movement Primitives})
pour apprendre (par démonstration) la forme du meilleur mouvement de tir connu.
Grâce aux propriétés des DMP, ce tir initialement non paramétrable est étendu 
à différentes positions possibles de la balle.
La stabilisation est assurée par un asservissement du ZMP (\textit{preview control}).\\

Sur un bras robotique, \cite{christiano_transfer_2016} corrigent les différences 
de prédiction entre la simulation et le comportement réel du robot par 
l'apprentissage d'un modèle dynamique inverse.

\cite{boeing_dynamic_2016} travaillent sur un tout petit robot humanoïde 
fabriqué avec des servomoteurs analogiques et incluant des pièces mécaniques flexibles. 
Leur but est d'optimiser le mouvement de marche du robot en simulation, en se basant
sur un algorithme génétique sans gradient.
Un modèle des servomoteurs est présenté afin de capturer au mieux la dynamique 
réelle du robot.
Le simulateur ouvert \textit{DynaMechs} est employé.

Les travaux de \cite{farchy_humanoid_2013} et \cite{hanna_grounded_2017}
sont les plus proches de ce que nous aimerions réaliser avec Sigmaban.
L'objectif des auteurs est d'améliorer la vitesse de marche d'un robot NAO.
Pour ce faire, les paramètres du mouvement sont optimisés sur un 
simulateur standard (SimSpark, ODE) avec l'algorithme sans gradient CMA-ES. 
Pour compenser les déviations existantes entre les prédictions du simulateur et le
comportement réel du robot, différentes méthodes de régression sont comparées 
dans le premier article. Dans le second article, des réseaux de neurones profonds 
sont employés.
Après plusieurs itérations d'optimisations dans le simulateur puis d'expérimentations sur
le robot réel et re-apprentissage de la correction du simulateur, la vitesse de marche
du robot est significativement augmentée.\\

On peut également mentionner ici les travaux très originaux 
de \cite{degrave_differentiable_nodate}.
Les auteurs développent (au prix de quelques simplifications et omissions) 
une simulation dynamique entièrement différentiable.
Sur un tel simulateur, l'optimisation d'un mouvement contrôlé par un réseau 
de neurones profond est rendue possible avec une adaptation de l'algorithme de 
rétropropagation du gradient.
Par rapport à l'utilisation d'un algorithme d'optimisation sans gradient, 
un gain très important de performances est alors théoriquement possible.\\

