
\section{Perspectives\label{sec:perspectives}}

\subsection{Imperfections des robots}

Dans la mesure du possible, il est très largement préférable 
de tenter de résoudre directement les imperfections matériels des robots
plutôt que de les compenser par des techniques logiciels.
Dans le cas de nos petits robots humanoïdes, plusieurs 
pistes peuvent être explorées :\\

Pour commencer, l'équipe a commencé à expérimenter en 2017 la fabrication
de certains des segments mécaniques des robots en fibres de carbone.
Ce matériau réunit de nombreux avantages.
Plus léger que l'aluminium, il est solide et ne se déforme pas avec le temps.
Par contre, il est cassant, ne peut donc pas être plié et son prix est plus élevé.
Malheureusement, les segments mécaniques ne sont pas les seuls à se déformer
(voir section \ref{sec:robot_flaws}).
Les arbres de sortie des servomoteurs tendent également à se tordre en torsion.
Pour eux, il n'y a pas réellement de solution envisageable, puisque qu'ils 
sont intimement liés aux servomoteurs Dynamixel.

Ces déformations mécaniques ont principalement lieu
pendant les matchs de football robotique,
lors des chutes et des chocs violents contre les autres robots,
alliés ou adverses.
Étant donné que sans perturbation extérieur, 
la stabilité de notre mouvement de marche est satisfaisante, 
une solution plus pérenne serait donc de limiter autant 
que possible les collisions avec les autres robots.
Une ébauche d'algorithme de détection et d'évitement d'obstacle
a été expérimenté par l'équipe lors de l'édition 2017 de la RoboCup.
De la même manière que cette fonctionnalité est imposée dans la ligne des
robots humanoïdes standard NAO, cette approche est certainement 
à développer pour sauvegarder la mécanique des robots.\\

Concernant les jeux mécaniques des servomoteurs, il reste à identifier
précisément leur origine en comparant précisément 
un moteurs neuf à un moteur usé.
Si le problème provient bien de l'usure des dents des engrenages, 
il n'y aura pas d'autre solution envisageable que de les remplacer 
à intervalles réguliers.
Si par contre, le support plastique des axes de rotations des
engrenages tend à se déformer, un boitier métallique réduirait
alors la vitesse d'usure.\\

Afin de réduire la résistance électrique présent sur le bus
d'alimentation série, il a été proposé de
directement souder les câbles du bus sur la carte électronique du servomoteur.
Cette approche possède malheureusement comme principal inconvénient 
d'augmenter l'effort humain de fabrication mais surtout de remplacement
des câbles\footnote{L'usure et les faux contacts électriques à l'intérieur
des câbles Dynamixel sont malheureusement assez courant. 
Leur remplacement est alors indispensable.}.
L'échange des connecteurs Dynamixel par d'autre connecteurs, 
plus robustes mais plus volumineux est également un compromis à tester.\\

Enfin, l'amélioration du suivi de trajectoire des servomoteurs
passe par l'ajout d'un contrôle plus riche en pré-compensation (\textit{feedforward}).
Soit directement au sein des servomoteurs comme expérimenté par Rémi Fabre
avec Dynaban (\cite{DynabanRoboCup2016}).
Soit de l'extérieur en biaisant volontairement la consigne de position ;
étudié récemment par Philipp Schlehuber.
La correction se base à la fois sur le modèle dynamique inverse du robot
mais également sur le modèle électrique des moteurs.
Ces modèles doivent donc être préalablement identifiés.

Il faut également noter que les toutes dernières versions du logiciel
de contrôle interne des servomoteurs Dynamixel ajoutent à leur tour une fonctionnalité
de pré-compensation.
Moins \og configurable \fg que l'utilisation de Dynaban, cette mise à jour
a l'avantage d'être dès aujourd'hui compatible avec les différents modèles
de servomoteurs\footnote{Nouvelle documentation 
\textit{MX-64} : 
\url{http://support.robotis.com/en/product/actuator/dynamixel/mx_series/mx-64(2.0).htm} 
et \textit{MX-106} : 
\url{http://support.robotis.com/en/product/actuator/dynamixel/mx_series/mx-106(2.0).htm}}.
Le support de ces nouvelles fonctionnalités reste à implémenter au sein de notre
structure logicielle.
L'utilisation de la pré-compensation dans tous les mouvements du robot 
va également nécessiter une modification de notre couche motrice.
En effet, pour compenser les délais, le mouvement désiré de chaque articulation
devra pouvoir être calculé légèrement dans le futur.

\subsection{Correction de l'odométrie}

Les effets des imperfections de nos petits robots humanoïdes
sur les odométries proprioceptive et prédictive
ont été étudiés et en partie corrigés aux sections 
\ref{sec:odometry_lwpr} et \ref{sec:odometry_cmaes}.
Au vu des résultats, les perspectives suivantes sont à envisager :\\

Tout d'abord dans la deuxième étude, il serait important de compléter 
l'analyse des performances des politiques de contrôle du mouvement de marche.
Ces politiques, soit représentées sous la forme d'une forêt d'arbres de régression,
soit manuellement implémentées et paramétrées, sont optimisées à partir 
de la simulation des déplacement du robot.
Malheureusement, les travaux proposés ne permettent pas de savoir si
la correction de l'odométrie prédictive est réellement importante
pour la qualité des politiques générées.
Est-ce que la prise en compte par la simulation du comportement réel du robot
améliore la pré-compensation des imperfections, 
ou est-ce que l'asservissement suffit à obtenir un contrôle robuste et précis ?\\

Ensuite, les résultats obtenus par \cite{oriolo_humanoid_2016} nous incitent
à expérimenter sur nos robots les bibliothèques disponibles d'odométrie visuelle.
Ceci afin de réellement tester leur temps de calcul en embarqué et également
d'appréhender leurs limitations quant à l'environnement visuelle.
La combinaison des deux méthodes, la correction de l'odométrie proprioceptive
par apprentissage et l'estimation de odométrie par analyse visuelle, 
pourrait alors être étudiée.
Le principal avantage de l'odométrie visuelle étant la mesure possible 
des glissements des pieds du robots ; mesure impossible à partir 
des capteurs proprioceptif disponible sur Sigmaban.\\

L'algorithme de maximisation de la (log) vraisemblance peut être
appliqué afin de raffiner l'apprentissage du modèle de correction 
linéaire des odométries proprioceptive et prédictive.
Dans la section \ref{sec:odometry_cmaes}, les paramètres du modèle de correction
sont optimisés afin de minimiser directement la distance entre le déplacement 
estimé au travers du modèle et le déplacement expérimentalement mesuré.
Cette optimisation ne prend pas en compte, ni les différentes sources
de bruit expérimental, ni le fait que les déplacements de plus
longue distance cartésienne sont alors surpondérés.
La vraisemblance est utilisée comme critère à maximiser dans 
la recherche des paramètres du modèle de correction.
Elle évalue la pertinence statistique des paramètres
d'un modèle au regard des données expérimentales.
Ce critère tend à réduire les cas de surapprentissage et
permet de plus d'identifier un modèle (également linéaire) du bruit
du déplacement (dépendant des ordres de la marche).

L'implémentation de cette méthode a été commencée
à l'occasion de l'édition 2017 du \textit{German Open} RoboCup.
Bien que qualitativement fonctionnelle, une étude statistique 
précise n'a pas encore été menée.
De plus pour des raisons pratiques, les séquences d'apprentissage ont été enregistrées 
en pilotant le robot manuellement et non pas de façon automatique et aléatoire.
Enfin, ni l'odométrie prédictive, ni les valeurs du modèle de bruit
non pour l'instant été analysés en détail.
À terme, l'objectif est de fournir une évaluation plus juste du bruit 
de déplacement aux méthodes de génération de politiques de contrôle
du mouvement de marche\footnote{Sous l'hypothèse que le modèle 
du déplacement prédictif du robot permet bien d'améliorer la performance 
des politiques générées en simulation.}.\\

Une autre extension possible de ces travaux est
l'amélioration du temps et de la simplicité d'enregistrement
des séquences de déplacements nécessaires à l'apprentissage.
L'équipe a par exemple débuté le développement
d'un système de localisation aisément transportable.
Son fonctionnement repose sur la détection par 
le robot de balises visuelles (QR code) disposées tout autour de 
la zone de marche sur des support léger en bois.
Même si la précision d'un tel système reste à évaluer,
la mise au point d'une procédure d'enregistrement intégralement 
automatisée semble réalisable.
À noter qu'avec ce système, la calibration du modèle de caméra
du robot serait aussi simultanément possible.\\

L'intégralité des expérimentations sur l'odométrie
réalisées ici sont basées sur le générateur de marche \textit{IKWalk}
utilisé par l'équipe de 2014 à 2016 (voir la section \ref{sec:walk}).
Le nouveau mouvement de marche \textit{QuinticWalk} n'a été développé que récemment.
Qualitativement, ce mouvement semble être bien plus stable, 
moins sujets aux oscillations parasites.
Il serait très intéressant d'appliquer l'apprentissage des odométries
prédictive et proprioceptive sur cette nouvelle marche puis de comparer
les paramètres de leur modèle de bruit respectif.
On peut faire l'hypothèse que la stabilité qualitative de la marche
\textit{QuinticWalk} sera visible au travers du bruit réduit de son déplacement.

Sur les petits robots quadrupèdes AIBO, les travaux de
\cite{rofer_evolutionary_2004} peuvent ici être soulignés.
Dans cet article, l'auteur optimise les paramètres de son générateur
de marche à l'aide d'un algorithme génétique sur le robot réel.
L'originalité de la démarche provient du fait que 
sa fonction de récompense prend en compte l'erreur entre
le déplacement désiré et le mouvement proprioceptif perçus.
Le mouvement de marche est ainsi optimiser de sorte à
réduire l'erreur de l'odométrie proprioceptive.
L'idée de l'étude du mouvement de marche au travers de 
la précision de son odométrie est intéressante et 
pourrait être appliquée à nos robots humanoïdes.

\subsection{Simulation et synthèse de mouvements}

À l'heure actuelle, une partie des implémentations logicielles sont achevées ;
synthèse de mouvement par optimisation et dynamique inverse, 
simulation (dynamique directe) du robot avec contact sans glissement,
ré-optimisation des mouvements avec le simulateur.
Néanmoins, il reste encore beaucoup à faire.
Notamment améliorer la qualité du simulateur et réaliser 
de nombreuses expérimentations afin de valider nos prédictions
sur des systèmes mécaniques de complexité croissantes.\\

La génération de mouvements \og optimaux \fg, par optimisation
d'un critère de performance basé sur la dynamique inverse 
du robot est un domaine déjà très étudié.
Dans la littérature, le modèle dynamique inverse ainsi que
le critère de performance sont différentiés.
Le calcul du gradient permet alors une recherche efficace 
des optimums locaux.
Dans l'état actuelle de notre implémentation, cette optimisation
repose sur une approche sans gradient (boite noire).
En terme de temps de calcul, cette méthode est bien moins 
efficace de plusieurs ordres de grandeurs.
Le point délicat à traiter est principalement la différentiation 
de la dynamique inverse dans le cas du double support.
Ceci à cause du fait que le problème admet une infinité de solutions.
Ce problème est par exemple résolu par \cite{lengagne_planification_2009}
ayant généré des mouvements sur le petit robot humanoïde HOAP-3.\\

L'amélioration du simulateur passe aussi par la gestion du glissement
des points de contact du robot avec le sol.
Typiquement avec l'implémentation de la méthode classique de 
\cite{anitescu_formulating_1997} et \cite{stewart_implicit_1996}
et l'utilisation de la dynamique impulsive.
Par exemple pendant le tir, il est possible que le léger glissement 
du pied de support au moment de l'accélération maximale 
du pied en vol joue un rôle significatif dans la stabilité du tir expert.\\

Enfin, l'identification des paramètres du simulateur 
est l'une des principales difficultés de la méthode.
Pour le moment, une approche simple par optimisation est utilisée ;
très semblable à ce qui a été fait pour l'odométrie.
Cependant, les temps de calculs sont ici très élevés et rendent 
cette approche difficile à mettre en oeuvre au regard 
du nombre important de paramètres recherchés.

Heureusement, le domaine de l'identification de modèles 
dynamiques possède une littérature très riche.
Son étude est nécessaire à la poursuite de ces travaux.
Sans nécessité de capteurs de force, on peut par exemple
mentionner les travaux de \cite{schwarz_compliant_2013} identifiant
les paramètres d'un modèle des servomoteurs par apprentissage itératif
du contrôle (\textit{iterative learning control}).\\

L'intérêt du simulateur est d'apporter autant d'a priori que possible 
au problème. Cependant, il subsistera toujours une erreur, 
plus ou moins grande entre le comportement réel est les prédictions dynamiques.
L'approche suivit dans les travaux récents de \cite{hanna_grounded_2017} 
et \cite{farchy_humanoid_2013} est particulièrement intéressante.
Au lieu de se reposer sur une étape d'identification complexe du simulateur,
des techniques de régressions non paramétriques (typiquement des réseaux de neurones profonds)
sont appliquées.
Plus précisément, les prédictions du simulateur ne sont pas \og corrigées \fg,
mais les ordres moteurs sont modifiés afin de produire les mêmes effets 
dans le simulateur que sur le robot réel.
L'expérimentation de ces méthodes semble très prometteuse.
Sachant l'évolution des imperfections de nos robots, il est très intéressant de combler 
la dernière étape de prédiction par des techniques d'apprentissage.
A priori, ces techniques seront d'autant plus efficaces et capables de 
généraliser à d'autre mouvements, que le simulateur de base 
calcul un comportement proche de la réalité.\\

La littérature propose d'expérimenter un moyen simple pour
réduire la différence entre le comportement en simulation et dans le
monde physique. Pour éviter que l'algorithme d'optimisation ne tire parti
des défauts d'implémentation du simulateur, l'ajout d'un bruit
tendrait à rendre la solution trouvée robuste aux variations de la dynamique du robot. 
En espérant que cette robustesse suffise a permettre le transfert dans le monde réel. 
Malheureusement, on peut faire l'hypothèse que cette solution tend à préférer 
des mouvements plus \og conservatifs \fg et donc moins performants.\\

Même si cela représente un investissement en temps significatif, 
la réimplémentation d'un simulateur dynamique permet d'avoir la maitrise complète 
des différents modèles (dynamiques, contacts, servomoteurs) mis en oeuvre.
De la même manière que pour la correction de l'odométrie, 
plusieurs niveaux de complexité sont envisageables pour chacun de ces modèles
(modélisation du frottement linéaire ou de Stribeck, jeux mécaniques, contacts avec
ou sans glissement, prise en compte des défauts électriques, inertie interne des servomoteurs).
Une des principales perspectives de cette implémentation tient en 
la quantification de l'apport de chaque modèle à la qualité de prédiction 
du comportement de nos petits robots humanoïdes.\\

La connaissance de plusieurs exemples \og positifs \fg de mouvements de tir
experts, stables et puissants, est un atout majeur pour la poursuite de ces travaux.
La première étape étant de réussir à les simuler ; c'est à dire à ce que
ces mouvements hautement dynamiques soit aussi bien stables expérimentalement
que dans la simulation.
L'analyse de leur stabilité en simulation permettrait alors peut être
de mieux comprendre l'effet des imperfections du robot sur la dynamique.
En effet car d'après notre modèle dynamique inverse du robot, 
ces mouvements ne devraient en théorie pas être stables.
À noter que la forme des mouvements de tir experts sont cependant
très différents de la forme paramétrée (voir section \ref{sec:motion_template}) 
du gabarit du mouvement de tir en double support utilisé lors de l'optimisation.
Une identification des paramètres à été commencé afin de \og convertir \fg
les mouvements experts dans la représentation utilisée à la section \ref{sec:motion_generation}.
Ainsi, les mouvements experts pourront être utilisés pour initialiser l'optimisation
et tenter de les améliorer automatiquement encore un peu plus.\\

Un des objectifs les plus ambitieux serait à terme la synthèse 
d'un mouvement de marche dynamique, stable sur le robot réel.
Le mouvement de marche est en effet particulièrement difficile à 
simuler. Comme le mouvement n'est pas épisodique mais cyclique, 
l'accumulation des erreurs avec le temps rend très difficile
la prédiction précise de la dynamique du robot sur le long terme.
De plus, la bonne simulation des contacts, des chocs et certainement 
également du glissement est un pré-requis.
À noter que \cite{hanna_grounded_2017} ont réussi sur un robot NAO
à améliorer significativement la stabilité et la vitesse de marche 
en s'appuyant sur une approche similaire.\\

S'il était possible de synthétiser en boucle ouverte un mouvement
de marche ainsi qu'un mouvement de tir en double support,
une autre étude très intéressant serait alors à mener.
L'analyse de l'enchainement des deux mouvements permettrait
de grandement accélérer le temps de tir.
Par exemple, pendant le mouvement de marche, quelle est la meilleure 
phase où déclencher le mouvement de tir. Puis générer alors 
un mouvement de tir démarrant directement à cet état géométrique 
et cinématique du robot.\\

En plus des mouvements en boucle ouverte,
l'étude de la génération de mouvements réactifs est également
une piste de recherche possible.
L'intérêt de l'approche par optimisation en boite noire sur le simulateur
est qu'aucune forme n'est imposée sur la politique de contrôle 
(le calcul du gradient de la commande en fonction des paramètres 
de la politique n'est pas nécessaire).
Par exemple, un problème intéressant et accessible est celui de la stabilisation 
du robot humanoïde immobile en simple support (debout sur un pied).
Face à une perturbation extérieure et ne pouvant plus exercer de couples 
sur le sol à cause du roulement, l'humain tend à utiliser 
son bassin comme masse de réaction afin de lutter contre le basculement.
D'une part, cette configuration en simple support est simple à étudier et à simuler.
D'autre part, ce problème est également très proche d'un des défis techniques 
(\textit{technical challenge}) faisant actuellement parti de la compétition RoboCup.\\

